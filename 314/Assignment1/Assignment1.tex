\documentclass[10pt,a4paper]{article}
\usepackage[latin1]{inputenc}
\usepackage{amsmath}
\usepackage{amsfonts}
\usepackage{amssymb}
\usepackage{subfig}
\usepackage{graphicx}
\usepackage[backend=bibtex]{biblatex}
\addbibresource{ref.bib}


\author{Regan Koopmans, 15043143}
\title{COS 314 - Assignment 1 }

\begin{document}
		
	\maketitle
		
		\section{Intelligence}
			\subsection{Origins}
				The word \textsl{intelligence} derives from the Latin word \textsl{intelligere}, meaning \textsl{to understand} \cite{etymonline2017}.
		
			\subsection{Meaning in Personal Opinion}
		
				In my opinion, intelligence is the ability for some agent to comprehend the significance of its actions in the external world, with some predictions of its effects. In other words, I believe that intelligence is synonymous with an \textsl{internal model of reality}. For illustration of this idea we can look at animals:
				\\\\
				Simple creatures, such as star-fish, survive completely on instinct. These animals can be described entirely by static physical and chemical processes. In this way, we can imagine these animals like any other natural system, such as the weather or rock formations. These animals are \textsl{alive}, but are not \textsl{intelligent}, and exist on predictable chemical rules. Contrast this to a crow, which has been shown to use tools to obtain food \cites{atlantic2017}. This behaviour is fundamentally different from that of the star-fish. The crow, like us, has the ability to see a desired outcome and design (possibly new) steps to achieve that goal. Specifically:
				\medskip
				\[
				\text{Physical outcome}\rightarrow\text{Access to food}\rightarrow\text{Survival}
				\]
				Unlike the star-fish, the actions of the crow are \textbf{motivated}, \textbf{calculated} and \textbf{premeditated}. This is the defining feature between the two animals. Collectively we may call this property \textsl{intelligence}. Other characteristics we attribute to intelligence (such as self-awareness, reasoning and complex decision-making) are easily explained by this essential difference. Indeed, it is precisely for this distinction that these higher-order mental activities become useful to a living creature. Simple creatures still thrive without intelligence, and hence stay simple.
				\\\\
				It is still a matter of debate whether intelligent creatures are deterministic in the same way bacteria and parasites are. It may be the case that complexity is the only factor that separates the star-fish from the crow. This links to questions of the existence of free-will and the true nature of conciousness. Such philosophical questions are beyond the scope of this discussion, but I believe that I have sufficiently illustrated why I feel that intelligence is so closely tied to an internal representation of reality.
				
		
		\pagebreak
		\section{Artificial Intelligence}
		 	\subsection{Origins}
				The term \textsl{Artificial Intelligence} was coined by John McCarthy in 1955. However the concept of ``thinking machines" is much older.
		
		
			\subsection{Meaning in Personal Opinion}
		
				
				
				As described in \textsc{Section 1 - Intelligence}, I believe that intelligence is best described as a sandbox model of reality that an agent can use to evaluate scenarios. Hence I am of the opinion that we can make machines intelligent in the same manner that we are, assuming a sufficient model. In most applications of Artificial Intelligence we have the following: 
				
				\begin{itemize}
					\item a value to optimize or a task to complete.
					\item features, or ways in which a strategy might be modified.
					\item feedback, in the form of some model (typically number of successes in environment).
				\end{itemize}
				
				With these conditions, and sufficient data/time to iterate, many tasks that we deem intelligent (image recognition, language-processing and language comprehension), have been effectively simulated on computers.
		
			\subsection{Main Purpose of The Field}
		
				Artificial Intelligence is primarily concerned with creating programs and machines that can reason with similar faculties as humans, in the aim of solving complex problems. 
				
				A bi-product of these studies is further understanding in the cognitive process in biological animals, such as humans. Artificial Intelligence can show us whether conciousness is an emergent property (whether self-awareness simply falls out of mechanical structures like the brain).
		
			\subsection{Contemporary Success}
				Modern Artificial Intelligence is characterised by the proliferation of computation speeds and data availability, which .
		
		\pagebreak
		\section{Life}
		
			\subsection{Origins}
		
				The word \textsl{life} comes from Old English. It is most likely Germanic in origin, and has kept its original meaning through history \cites{etymonline2017}. Life is a fundamental concept for mankind (for obvious reasons), and features heavily in religion, ceremony and human experience.
	
			\subsection{Meaning in Personal Opinion}
	
				There are formal biological definitions of life. However in my opinion, life as we know it is an abstract concept describing regular patterns in carbon-based molecules. As minerals and crystals are the regular forms of atoms, so life (in its denominations) is the regular form of primordial carbon substances. Living things take in raw materials, and use these to build regular structures, as prescribed by DNA, and to spawn new structures. 
				\\\\
				This. The diagram below perhaps illustrates this point better:
				\\\\
				\begin{figure}[h]
					\centering
					\subfloat[Shelled sea animals \cite{Sea}]{{\includegraphics[width=4cm]{animals} }}%
					\qquad
					\subfloat[Crystals \cite{Crystal}]{{\includegraphics[width=5cm]{crystals} }}%
					\caption{Similarities of crystalline and biological structures.}%
					\label{fig:example}%
				\end{figure}
				
		
		\pagebreak
		\section{Artificial Life}
		
			\subsection{Origins}
		
				The term \textsl{Artificial Life} was coined by Christopher Langton in 1986. It describes a field of study that researches the possibility, whether it be robotic, bio-chemical or completely simulated in software.
		
			\subsection{Meaning in Personal Opinion}
		
				In my opinion. I believe that living beings represent a repetitive structures through time.
		
			\subsection{Main Purpose of Field}
			
				Artificial Life
				
				
			
			\subsection{Contemporary Success}
			
				
					
		\pagebreak
		\printbibliography

\end{document}